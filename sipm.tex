\documentclass{beamer}
\usetheme{Berkeley}

\title{The silicon photomultiplier}
\author{Stefan Gundacker and Arjan Heering}
  
\institute{Presented: by Yannick Prianon,\\ Jonas Berggren, Anna Dunz}

\begin{document}

%what is SiPM
%how does it work: parameters

%quality quantities
  %equation: SPTR, PDE
%errors

%analog vs. digital
%application
%prospect

\begin{frame}
\titlepage
\end{frame}

\section{P.O.O.}
\begin{frame}
\frametitle{SPAD: Principles of operation}
  \begin{itemize}
    \item p-n diode
    \item 3 regimes:
    \begin{enumerate}
      \item simple e-h creation through impact ionization
      \item secondary ionization by e
      \item secondary ionization by e and h (quenching needed)
    \end{enumerate}
    \item e ionization is more sensitive
    \item n-on-p for red
    \item p-on-n for blue
  \end{itemize}
  %TODO: fig 1, 2, 3
\end{frame}

\begin{frame}
  \frametitle{electrical equivalent circuit of the SPAD}
  \begin{itemize}
    \item 
  \end{itemize}
\end{frame}

\section{Parameters}
\begin{frame}
  \frametitle{Breakdown voltage and multiplication gain}
  \begin{itemize}
    \item breakdown voltage
      \[\int_0^W\alpha_ne^{-\int_0^x(\alpha_n-\alpha_p)dx'}dx=1\]
    \item $V_{bd}$ is more advantageous for thin depletion regions
    \item
      \[Gain=\frac{avalanche\_charge}{q}=\frac{V_{ov}(C_q+C_d)}{q}\]
    \item gain gives signal well above noise for $q=e$
  \end{itemize}
\end{frame}

\begin{frame}
  \frametitle{$V_{bd}$}
  \begin{itemize}
    \item measuring the gain as a function of $V_{bias}$.
    \item find the derivative
  \end{itemize}
  %sec 3.2
\end{frame}

\begin{frame}
  \frametitle{temperature dependence of parameters}
  \begin{itemize}
    \item $V_{bd}$ increases with $T$
    \item gain increases with lower $T$
    \item DCR increases with $T$
  \end{itemize}
\end{frame}


\section{Quality measures}
\begin{frame}
  \frametitle{PDE}
  \begin{itemize}
    \item 
      \[PDE(V_{ov},\lambda)=QE(\lambda)P_T(V_{ov},\lambda)FF_{eff}(V_{ov},\lambda)\]
  \end{itemize}
  %eq. 12
\end{frame}

\begin{frame}
  \frametitle{SPTR}
  \begin{itemize}
    \item 
      \[\sigma_{timing}=\frac{\sigma_{v_{voise}}}{dv/dt_{@threshhold}}\]
  \end{itemize}
\end{frame}

\section{error sources}
\begin{frame}
  \frametitle{error sources}
  \begin{itemize}
    \item DCR
    \begin{itemize}
      \item $\propto\frac{1}{T}$
      \item residue at cryo. temperature: trap assisted tunneling
    \end{itemize}
    \item after pulse:
    \begin{itemize}
      \item release of trapped charges in high field regions
      \item mitigation: slow recharge rate
      \item secondary photons
      \item mitigation: low life time substrate
    \end{itemize}
  \end{itemize}
  %fig 5
\end{frame}

\begin{frame}
  \frametitle{Crosstalk}
  \begin{itemize}
    \item photons arising from one cell triggering a signal in an other cell
    \item Mitigation: optical trenches
    \item delay cross talk
    \item Mitigation: reduce undepleted region thickness
    \item external crosstalk: emission of secondary photons producing reflection on "the window"
    \item Delayed vs. prompt cross talk.
  \end{itemize}
\end{frame}

\begin{frame}
  \frametitle{Excess noise factor}
  \begin{itemize}
    \item Expectation: $N_{phe}\pm\sqrt{N_{phe}}$
    \item we define
      \[ENF=\frac{(\sigma_Q/<Q>)^2}{(\sigma_{Q_N}/<Q_N>)^2}\]
  \end{itemize}
\end{frame}

\begin{frame}
  \frametitle{Saturation}
  \begin{itemize}
    \item trade of between saturation and dead space
      \[N_{fired}=N_{total}\left(1-e^{-\frac{N_{photon}PDE\cdot ENF}{N_{total}}}\right)\]
  \end{itemize}
\end{frame}

\section{Digital SiPM}

%Jonas
\begin{frame}
  \frametitle{digital SiPM}
  \begin{itemize}
    \item each cell has it's own read out
    \item high data production
    \item compromise: small SiPM arrays
  \end{itemize}
\end{frame}

\section{Applications}
%Jonas
\begin{frame}
  \frametitle{Applications}
  \begin{itemize}
    \item TOF-PET
    \item PET-MR
    \item SPECT/MR
    \item HEP
      \begin{itemize}
        \item Problem: High radiation environment
      \end{itemize}
    \item LIDAR
    \item Scintillation detection
      \begin{itemize}
        \item Cherenkov in scintillator
      \end{itemize}
  \end{itemize}
\end{frame}

\section{Prospect}
%Jonas
\begin{frame}
  \frametitle{Prospect}
  \begin{itemize}
    \item Digital will become more prevalent
  \end{itemize}
\end{frame}


































%\begin{frame}
%  \frametitle{}
%  \begin{itemize}
%    \item 
%  \end{itemize}
%\end{frame}






\begin{frame}
\frametitle{Abbreviations}
  \begin{itemize}
    \item SiPM: Silicon photomultiplier
    \item SSPM: solid state photmultiplier
    \item MPPC: multi pixel photon counter
    \item SPAD: single photon avalanche diode
    \item SPTR: single photon time resolution
    \item PDE: photon detector efficiency
    \item PMT: photo multiplier tube
    \item TCAD: technology computer aided design
    \item DCR: dark count rate
    \item LTE: light transfer efficiency
    \item PTS: photon transfer time spread
  \end{itemize}
\end{frame}



\begin{frame}
  \frametitle{Abstract}
  \begin{itemize}
    \item What is SiPM and what is it use for
    \item What are relevant qualitative parameters for best application specific performance
  \end{itemize}
\end{frame}

\begin{frame}
\frametitle{introduction}
  \begin{itemize}
    \item Analog vs. Digital SiPM
    \item $10-100\mu m$ wide
    \item single photon sensitivity
    \item ps timing resolution
    \item applications:
    \begin{itemize}
      \item TOF-PET
      \item LIDAR
      \item dark matter
      \item double $\beta$
      \item HEP
      \item much more
    \end{itemize}
  \end{itemize}
  %fig 6
\end{frame}

\begin{frame}
  \frametitle{electronic read out}
  \begin{itemize}
    \item high gain
    \item high power consumption
    \item strong and fast amplifier
  \end{itemize}
  %sec. 3.9
\end{frame}

\begin{frame}
  \frametitle{Analog SiPM}
  \begin{itemize}
    \item large array of SPADS in parallel
    \item Counting photons via integration over charge
    \item individual SPADS interact via capacitance
  \end{itemize}
  %sec. 3.1
  %eq. 3, 4?
\end{frame}
\end{document}
